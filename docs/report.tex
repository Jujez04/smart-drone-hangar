\documentclass[12pt,a4paper]{article}

% --- PACCHETTI ---
\usepackage[utf8]{inputenc}
\usepackage[italian]{babel}
\usepackage{geometry}
\usepackage{graphicx}
\usepackage{tikz}        % Per i diagrammi FSM
\usepackage{float}       % Per posizionare le immagini
\usepackage{array}       % Per tabelle avanzate
\usepackage{booktabs}    % Per tabelle estetiche
\usepackage{listings}    % Per il codice sorgente
\usepackage{xcolor}      % Per i colori
\usepackage{hyperref}    % Per i link e l'indice interattivo
\usepackage{bookmark}    % Per aggiornare automaticamente gli outline
\usepackage{longtable}   % Per tabelle lunghe su più pagine

% --- LIBRERIE TIKZ ---
\usetikzlibrary{shapes,arrows,positioning,automata,calc,shadows}

% --- IMPOSTAZIONI PAGINA ---
\geometry{margin=2.5cm}

% --- STILE CODICE ---
\definecolor{codegreen}{rgb}{0,0.6,0}
\definecolor{codegray}{rgb}{0.5,0.5,0.5}
\definecolor{codepurple}{rgb}{0.58,0,0.82}
\definecolor{backcolour}{rgb}{0.95,0.95,0.92}

\lstdefinestyle{mystyle}{
    backgroundcolor=\color{backcolour},   
    commentstyle=\color{codegreen},
    keywordstyle=\color{magenta},
    numberstyle=\tiny\color{codegray},
    stringstyle=\color{codepurple},
    basicstyle=\ttfamily\footnotesize,
    breakatwhitespace=false,         
    breaklines=true,                 
    captionpos=b, 
    keepspaces=true,                 
    numbers=left,                    
    numbersep=5pt,                  
    showspaces=false,                
    showstringspaces=false,
    showtabs=false,                  
    tabsize=2
}
\lstset{style=mystyle}

\title{
    \vspace{-2cm}
    \vspace{1cm}
    \textbf{\Huge Smart Drone Hangar System}\\[0.5cm]
    \Large Assignment \#02 - Embedded Systems \& IoT\\[0.5cm]
    \large Academic Year 2025-2026
}
\author{
    \textbf{Justin Carideo} \\
    ID: 0001115610
}

\begin{document}

\maketitle
\thispagestyle{empty}

\newpage
\tableofcontents
\newpage

% ==========================================
% SEZIONE 1: INTRODUZIONE
% ==========================================
\section{Introduzione}

\subsection{Overwiew}
L'obiettivo è quello di creare un sistema embedded che rappresenta un "Smart Drone Hangar". Il sistema è composto da due sottosistemi:
\begin{enumerate}
    \item \textbf{Drone Hangar (Arduino)}: l'hangar intelligente che gestisce l'apertura/chiusura della porta, monitora la presenza del drone, la distanza e la temperatura interna, e visualizza lo stato su un display LCD.
    \item \textbf{Drone Remote Unit (PC)}: un'unità remota che consente di monitorare e controllare 
    l'hangar tramite una GUI Java. Esegue due operazioni in particolare: Takeoff(apertura) e Land(chiusura).
\end{enumerate}

\subsection{Componenti usate}
Il sistema utilizza le seguenti componenti hardware:
\begin{itemize}
    \item \textbf{Sensors}: HC-SR04 (Sonar), PIR, TMP36 (Temperature Sensor), Button
    \item \textbf{Actuators}: Servo Motor, LEDs (Red, Green), I2C LCD Display
\end{itemize}

% ==========================================
% SEZIONE 2: ARCHITETTURA
% ==========================================
\section{Architettura di sistema}

\subsection{Software Design Pattern}
Il software si basa su una architettura \textbf{task-based}, nella quale uno scheduler cooperativo gestisce le principali task del sistema.
In particolare possiamo trovare i seguenti layer che compongono il sistema:

\begin{itemize}
    \item \textbf{HWPlatform}: incapsula tutte le compoenenti hardware, fornendo un livello di astrazione che gestisce interazioni a basso livello.
    \item \textbf{Context}: una struttura dati condivisa tra le varie task del sistema.
    \item \textbf{Task Logic}: Il comportamento del sistema è distribuito su diverse task specializzate pianificate dallo Scheduler.
\end{itemize}

\subsection{Task trattate}
La FSM e le funzioni ausiliarie sono divise in task specifiche:

\begin{description}
    \item[HangarTask]
    Adotta il pattern \texttt{Multi-state task}. Incorpora gli stati principali e coordina le altre task aggiornando il \texttt{Context} condiviso.

    \item[DoorTask]
    Responsabile della gestione degli stati della porta dell'hangar (Servo Motore) in base ai comandi presenti nel Context.

    \item[AlarmTask]
    Monitora continuamente i limiti di temperatura e gestisce gli stati di pre-allarme e allarme, sovrascrivendo le operazioni normali quando necessario.

    \item[DistanceDetectorTask]
    Gestisce il Rilevatore di Distanza del Drone (DDD/Sonar). Si occupa dell'attivazione (triggering) e del filtraggio delle letture ultrasoniche per aggiornare la distanza nel Context.

    \item[PresenceDetectorTask]
    Gestisce il Rilevatore di Presenza del Drone (DPD/PIR). Si occupa della calibrazione e della sincronizzazione del rilevamento di presenza.

    \item[DisplayTask]
    Gestisce il monitor LCD I2C. Legge lo stato del sistema dal Context e aggiorna i messaggi visualizzati in modo asincrono.

    \item[SerialCommTask]
    Gestisce la comunicazione bidirezionale con la Drone Remote Unit (DRU) su PC, effettuando il parsing dei comandi in ingresso e inviando aggiornamenti di stato.

    \item[TestHWTask]
    Una task dedicata utilizzata durante la fase di sviluppo per verificare e calibrare le singole componenti hardware in isolamento.
\end{description}

% ==========================================
% SEZIONE 3: MACCHINA A STATI (FSM)
% ==========================================
\section{Finite State Machines}

\begin{figure}[H]
    \centering
        \includegraphics[width=0.8\textwidth]{img/hangarTaskFSM.png}
    \caption{FSM di HangarTask}
\end{figure}

\begin{figure}[H]
    \centering
        \includegraphics[width=0.8\textwidth]{img/dddTaskFSM.png}
    \caption{FSM di DistanceDetectorTask}
\end{figure}


\begin{figure}[H]
    \centering
        \includegraphics[width=0.8\textwidth]{img/doorTaskFSM.png}
    \caption{FSM di DoorTask}
\end{figure}

\begin{figure}[H]
    \centering
        \includegraphics[width=0.8\textwidth]{img/blinkingTaskFSM.png}
    \caption{FSM di BlinkingTask}
\end{figure}


\begin{figure}[H]
    \centering
        \includegraphics[width=0.8\textwidth]{img/dpdTaskFSM.png}
    \caption{FSM di PresenceDetectorTask}
\end{figure}


\begin{figure}[H]
    \centering
        \includegraphics[width=0.8\textwidth]{img/serialTaskFSM.png}
    \caption{FSM di SerialCommTask}
\end{figure}


\begin{figure}[H]
    \centering
        \includegraphics[width=0.8\textwidth]{img/alarmTaskFSM.png}
    \caption{FSM di AlarmTask}
\end{figure}


% ==========================================
% SEZIONE 4: HARDWARE
% ==========================================
\section{Implementazione Hardware}

\subsection{Schematic / Wiring}
 \begin{figure}[H]
    \centering
        \includegraphics[width=0.8\textwidth]{img/schematic.png}
    \caption{Configurazione Hardware del Sistema}
\end{figure}

\subsection{Configurazione Pin}
\begin{table}[H]
\centering
\begin{tabular}{|l|c|l|}
\hline
\textbf{Component} & \textbf{Pin (Arduino)} & \textbf{Type (Input/Output/PWM/Analog)} \\
\hline
Sonar TRIG & 9 & Output \\
\hline
Sonar ECHO & 10 & Input \\
\hline
PIR Sensor & 2 & Input \\
\hline
Servo Motor & 11 & PWM \\
\hline
Temp Sensor & A0 & Analog \\
\hline
LED Green 1 & 4 & Output \\
\hline
LED Green 2 & 5 & Output \\
\hline
LED Red & 6 & Output \\
\hline
Button & 3 & Input \\
\hline
I2C LCD & A4/A5 & Output \\
\hline
\end{tabular}
\caption{System Pinout}
\end{table}

% ==========================================
% SEZIONE 5: DRONE REMOTE UNIT (PC)
% ==========================================
\section{Drone Remote Unit (Java)}

\subsection{GUI}
 \begin{figure}[H]
    \centering
        \includegraphics[width=0.8\textwidth]{img/gui.png}
    \caption{A sinistra: Log View, a destra: Dashboard View}
\end{figure}

\subsection{Protocollo di comunicazione seriale tra Arduino e PC}

Il sistema utilizza un protocollo testuale basato su prefissi per distinguere la tipologia di informazione scambiata sulla porta seriale:

\begin{itemize}
    \item \textbf{\texttt{dh:}} (Drone Hangar): Messaggi di stato applicativi inviati da Arduino verso la GUI.
    \item \textbf{\texttt{lo:}} (Log): Messaggi di debug e informazioni di sistema per il monitoraggio interno.
    \item \textbf{\texttt{ru:}} (Remote Unit): Comandi inviati dalla GUI Java verso Arduino.
\end{itemize}

\vspace{0.5cm}

\begin{longtable}{@{} l p{6cm} l @{}}
\caption{Dettaglio del Protocollo di Messaggistica} \label{tab:protocollo} \\
\toprule
\textbf{Prefisso} & \textbf{Messaggio / Comando} & \textbf{Descrizione} \\
\midrule
\endfirsthead
\toprule
\textbf{Prefisso} & \textbf{Messaggio / Comando} & \textbf{Descrizione} \\
\midrule
\endhead
\bottomrule
\endfoot

\texttt{dh:} & \texttt{STATUS:DRONE\_INSIDE} & Il drone è all'interno dell'hangar. \\
\texttt{dh:} & \texttt{STATUS:TAKING\_OFF}  & Fase di decollo (apertura porta). \\
\texttt{dh:} & \texttt{STATUS:DRONE\_OUT}    & Il drone è fuori dall'hangar. \\
\texttt{dh:} & \texttt{STATUS:LANDING}      & Fase di atterraggio (chiusura porta). \\
\texttt{dh:} & \texttt{STATUS:ALARM}        & Stato di allarme critico (sovratemperatura). \\
\texttt{dh:} & \texttt{DIST:<valore>}       & Invio della distanza rilevata dal Sonar (cm). \\
\midrule
\texttt{lo:} & \texttt{System initialized}  & Notifica di avvio completato. \\
\texttt{lo:} & \texttt{CMD TAKEOFF}         & Log di ricezione comando di decollo. \\
\texttt{lo:} & \texttt{[TaskName] State:X}  & Debug dello stato interno di una specifica task. \\
\texttt{lo:} & \texttt{ERR CMD:UNKNOWN}     & Errore nel parsing del comando ricevuto. \\
\midrule
\texttt{ru:} & \texttt{TAKEOFF}             & Richiesta di apertura porta per decollo. \\
\texttt{ru:} & \texttt{LANDING}             & Richiesta di chiusura porta per atterraggio. \\
\texttt{ru:} & \texttt{RESET}               & Ripristino del sistema dopo un allarme. \\
\bottomrule
\end{longtable}

\end{document}